DETECTION AND MITIGATION OF LOW RATE DENIAL OF SERVICE ATTACKS: A SURVEY

1. Seven Ws about the survey example (Who, What, When, Where, Why, for Whom, hoW): Well established authors, mostly Member or Senior IEEE members.

2. Essence (it can extremely difficult to give entire essence in only one sentence): Summarizes and complements previous studies and surveys related to this specific type of attack.

3. Structure:
A. Propose a taxonomy of the LDoS attacks, (divided into three
broad categories based on their modus operandi: QoS attacks, Slow rate attacks, and Service queue attacks)
B. Detail numerous detection mechanisms and counter-measures available against eight types of LDoS
attacks (i.e. Throttle)
C.Provide a feature comparison table for some existing attack tools. 

4. Some relevant details: Focuses on Low Rate DoS attacks.

5. Example (here one can call a figure that explains an example using a pseudo-code;  ideally, the same application case should be used for all surveyed examples): Lots of Figures and Tables useful for attack patterns and taxonomy

6. Pros and cons:
PRO Aims at providing an extensive review of the literature for helping researchers and network administrators find up-to-date knowledge on LDoS attack.
PROS: Table and taxonomy very useful
PROS: Seems to have brought together all of the existing research and develop all the tables and figures for quick reference.
CONS: doesn't provide suggestion on future areas of work.

7. My opinion of this example and its potentials: Useful for helping researchers and network administrators find up-to-date knowledge on LDoS attacks. 
CON: doesn't provide suggestion on future areas of work.