DISTRIBUTED DDOS AND DEFENSES IN IOT (CLOUD)

1. Seven Ws about the survey example (Who, What, When, Where, Why, for Whom, hoW): Reputable journal, university and authors.

2. Essence (it can extremely difficult to give entire essence in only one sentence): Focuses on  motivations and reasons for attackers to select non-legacy IoT devices to launch DDoS attacks.

3. Structure: Introduction, Motivations for attackers to use non-legacy IoT devices, single vs multi vector attack patterns, taxonomy, defense mechanisms, suggestions.

4. Some relevant details: Focuses on  motivations and reasons for attackers to select non-legacy IoT devices to launch DDoS attacks.

5. Example (here one can call a figure that explains an example using a pseudo-code;  ideally, the same application case should be used for all surveyed examples): Fig. 2 current form of multi-vector DDoS attacks.

6. Pros and cons:
PRO A complete survey of DDoS attacks for both IoT and the cloud environment which was not present in the current literature.
PRO Ties research to Mirai/Reaper Botnet which used 148,000 infected IoT devices.
PRO DDoS Cloud Classifications


7. My opinion of this example and its potentials: Does a good job bringing together two areas with lots of research that hadn't been presented together before (IoT and cloud). Very comprehensive. Provides suggestions on how to defend.
CON: didn't really list areas for future research.