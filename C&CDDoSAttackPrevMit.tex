A SURVEY OF DDoS ATTACK, PREVENTION, AND MITIGATION TECHNIQUES

1. Seven Ws about the survey example (Who, What, When, Where, Why, for Whom, hoW): Good authors and publication.


2. Essence (it can extremely difficult to give entire essence in only one sentence): The authors present a comprehensive survey of DDoS attack, prevention, and mitigation techniques.

3. Structure: Introduction, Attack Targets and Motivations, Attack Strategies, Attack Phases (scanning, propagation, attack attempt), attack mechanisms, attack types, and attack on nontraditional systems (clouds, smart grids, smart homes, CPSs, and IoT systems). Recommended other areas of research.

4. Some relevant details: A response to the 2016 uptick in DDoS attacks.

5. Example (here one can call a figure that explains an example using a pseudo-code;  ideally, the same application case should be used for all surveyed examples): Figure 1 shows volume sizes in DDoS attacks from 2007-2016.

6. Pros and cons: 
PRO They cover nearly (if not all) DDoS attack types known at the time of the writing (2017). 
PRO Really good future suggestions for research:
1.  All recent major DDoS attacks are based on IoT botnets- Prevent creation of the IoT botnets, detection, and rejection of the flows from unsophisticated IoT devices (such as security camera, smart refrigerator, home routers).
2. Ensure the lowest possible consumption of the victim’s resources by the defense mechanisms while fighting against DDoS.
3. Scalabiilty: It is very important to test the real-life performance of those researches.
4. Ensure defense against Zero days.

7. My opinion of this example and its potentials: Very good paper, no wonder why it is top cited. Directions for other research are very good.